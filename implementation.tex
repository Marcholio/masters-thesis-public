\chapter{Implementation}
\label{chapter:implementation}

Three solutions out of all proposals were selected for the implementation phase: pair working combined with rotating triage responsibility, project presentations
and a README template. Initial considerations about practicalities were discussed with the COO and CCO to assign responsibilities and agree about the practical
implementation of each solution.
This chapter presents these initial implementation plans together with adjustments done during the implementation phase.

\section{Triage responsibility and pair working}

Rotating triage responsibility combined with pair working was considered the main solution and it also required the most preparations as it required careful planning to succeed.
The very first preparation for the triage process was to double the amount of Freshdesk licenses from five to ten. This allows sufficient amount of people to participate
in the triage process, although five of the licenses were assigned for part-time employees. After increasing the amount of licenses, a need for guidelines for both Freshdesk
triage principles and pair working practices was quickly identified as there was no structured instructions on how Freshdesk was
actually being used and how it was meant to be used. Previously the practices had been tacit knowledge and based on a mutual agreement between Freshdesk users. However, as the
amount of Freshdesk licenses was doubled, this was no longer scalable and there was a need for explicit instructions. It was agreed that it would be a
responsibility of the COO, as he had a clear understanding of the Freshdesk process and its relationship with billing for example. The initial instructions were also reviewed
by other employees and added as a part of QOCO playbook, which contains general instructions about organizational practices.

In addition to the Freshdesk usage instructions, it was also noted that a pair working guideline is required as everyone intended to participate in pair working
was not familiar with detailed pair working practices beforehand. This was assigned as a responsibility of the researcher as he was already familiar with pair
programming literature. It was also noted that pair programming is a well known practice and there are plenty of good guidelines available online, which could be 
utilized as a reference material. After a brief review on available materials, a GitHub gist by \citet{Sarrafieh2017} was selected as a template to be
customized for QOCO's purposes.

It was also identified that writing instructions about the usage of Freshdesk and pair working was not enough to succeed in adoption of new practices. As the plan was to change
the person responsible for triage on a daily basis, selecting the responsible for each day required a well planned procedure. To reduce change resistance and encourage
independent participation, it was considered a good solution to let employees reserve their triage turns with Google calendar. As Google calendar was already used for
vacation planning and other daily activities, it was a natural tool to be used for triage turns as well. In addition to the calendar, also a dedicated Slack channel was created
to discuss about triage related matters and gather feedback. Another convenient feature with a dedicated channel was an integration to Google calendar so that it would
automatically report who is in charge of triage for the day and additionally report a summary about oncoming week on Monday mornings. In addition to these reminders, it was
noted that there should still be someone overseeing the process and ensuring that each day actually has a responsible assigned. This was agreed to be responsibility of 
the researcher and he would check every Wednesday that next week's turns are assigned, and if they are not, remind people about it.

Shortly after the initial launch of the triage process, concerns about noticing tickets were raised. It was considered that it would be the easiest if Freshdesk would notify
the daily responsible automatically about newly created tickets so that the responsible would not need to constantly check for updates. As Freshdesk did not support this
functionality out of the box, a separate Slack channel for ticket notifications was created. The idea was that at the beginning of the shift, the new responsible joins the
channel and reviews the tickets that have been created after the last responsible left the channel. Then Freshdesk would continue to post notifications about newly created
tickets to the channel during the day and the responsible does not need to check Freshdesk itself frequently. Additionally as Slack supports a wide range of emoji reactions
to messages, it was also agreed that the daily responsible should put a check mark on ticket notifications after they have been handled. This way it is easier for the next
person to see which tickets are still waiting for triaging.

It was also evident already during the first days of the triage process that SLA limits and definitions for different priority levels required clarification. These clarifications
were added to QOCO playbook together with other triage guidelines. In addition to that, also a response template was added to Freshdesk describing resolution times for different
priority levels. Its purpose was to remind the customers about agreed resolution times, since it was considered that they were partly unclear. The intended use for the response template
was to be automatically added to the first response on new tickets and it could be left out in consecutive messages.

\section{README template}

README template did not require as much preparations as its use case was much more straightforward. The first version of the template was created to GitHub by the researcher
after studying the literature and several example templates available online. There was no single template that would have acted as an initial starting point to be adapted to
QOCO's context as quite many of the example templates were aimed for public open source projects rather than acting as an informal internal project documentation for some
company. Therefore it was decided that the template is created from scratch based on general headlines from the literature and open source templates. The first version
of the template is presented in appendix \ref{appendix:readme}. As said, it is the first version and it is meant to be updated if there are mistakes or things to be added to it
later on. The template was then explained in a company wide Slack channel and its usage was encouraged in projects under active development.

In addition to the template itself, it was also considered useful to create an example use case of it to some recent project. There was one project that had recently entered
the maintenance phase and was considered a useful example case. Therefore it was agreed that the researcher will update its README to match the newly created template.
This was also a good evaluation for the template as using it in an actual case could point out some missing features that could not be noticed while creating dummy content.

\section{Project presentations}

There were a few possible preparations that could have been done to decrease the effort required to prepare the project presentations. At first it was discussed that some
slide show template could potentially lower the barrier of preparing presentations as it would have all the necessary points already listed. However it was also pointed out
that the same effect could be achieved with a plain checklist as it could be difficult to create a useful slide show template to serve various use cases. It was agreed
that plain checklists for different use cases would be enough as a starting point and slide show templates could be added later on, if considered necessary. The checklist
was added to QOCO playbook so that it would be easily accessible and updatable by everyone. In addition a shared folder in Google Drive was created to store the presentation
materials.

There was also an idea about having presentations in different phases of a project and therefore it was agreed that it would be beneficial to try out presentations with
projects in different phases to gather feedback about which presentations were considered useful and which not. The initial plan was to have at least four presentations:
the first production deployment of one project, the newest features of a SaaS project, the business case of a relatively new project and the initial plan of an upcoming project.
To ease feedback gathering, a simple feedback form was created with Google Forms with a simple scale from zero to five as the only mandatory question and a voluntary
open comment field.