\chapter{Survey structure}
\label{appendix:survey}

\begin{center}

This is the structure for the knowledge sharing survey conducted at the beginning of February and at the end of April.
The survey was created with Google Forms.

\vspace{12pt}

\textbf{Notation:} \\
\textbf{This is a question} \\
\emph{This is an additional text from the survey} \\
These are clarifications for documenting purposes

\vspace{12pt}

\textbf{What is your role?} \\
\emph{Select the most suitable one}

\begin{itemize}
	\itemsep0em % Item separation
	\item \begin{center} \emph{Junior developer} \end{center}
	\item \begin{center} \emph{Senior developer} \end{center}
	\item \begin{center} \emph{Junior consultant} \end{center}
	\item \begin{center} \emph{Senior consultant} \end{center}
	\item \begin{center} \emph{Executive} \end{center}
\end{itemize}

\textbf{Work experience in years at QOCO?} \\
Integer scale from 0 to 10

\vspace{12pt}

\textbf{How familiar are you with these projects?} \\
List of 34 projects in random order, each evaluated with a 6-point scale:

\begin{enumerate}
	\itemsep0em % Item separation
	\setcounter{enumi}{-1}
	\item \begin{center} \emph{I do not know the business case} \end{center}
	\item \begin{center} \emph{I am familiar with the business case} \end{center}
	\item \begin{center} \emph{I am familiar with the business case and general architecture} \end{center}
	\item \begin{center} \emph{I can do some minor tasks} \end{center}
	\item \begin{center} \emph{I can do major tasks} \end{center}
	\item \begin{center} \emph{I know this project in detail} \end{center}
\end{enumerate}

\vspace{12pt}

\textbf{Do you think following barriers are a challenge at QOCO?} \\
\emph{Knowledge sharing barriers are things that disturb or prevent sharing of knowledge. Think especially in the context of QOCO Systems.}

\vspace{12pt}

5-step Likert scale (\emph{Strongly disagree, Disagree, Neither agree or disagree, Agree, Strongly agree}) was used to evaluate 20 knowledge sharing barriers in random order:

\vspace{12pt}

\emph{Lack of formal documentation\\
Lack of informal documentation\\
Lack of comments in code\\
Lack of trust\\
Messy and complex code\\
Lack of informal communication\\
Knowledge hoarding\\
Difference in experience levels\\
Tight schedule\\
Lack of motivation\\
Used communication tools\\
Multitasking\\
Complex domain (aviation industry)\\
Working in different locations\\
Difference in age\\
Individual communication skills\\
Strong code ownership\\
Physical work environment\\
Organization culture\\
Strong organizational structure}

\vspace{12pt}

\end{center}

\chapter{README template}
\label{appendix:readme}

\begin{tiny}
	\begin{verbatim}

# qoco-readme

### Table of Contents  
- [qoco-readme](#qoco-readme)
    - [Table of Contents](#table-of-contents)
  - [Business case](#business-case)
  - [Development](#development)
    - [Prerequisites](#prerequisites)
    - [Setting up local development environment](#setting-up-local-development-environment)
    - [Building](#building)
    - [Running or Usage](#running-or-usage)
    - [Deploying](#deploying)
  - [Environments](#environments)
  - [Support](#support)
  - [Related / dependent projects](#related--dependent-projects)
  - [Known limitations](#known-limitations)


## Business case

The business case of the project is briefly described here. What does this project do? Why does it exist?

This README template is meant to improve informal documentation of projects. It is meant to be used as a reference material about
necessary information required for README. It can be modified to fit the context of different projects, but the template itself
should be kept on a general level to ensure suitability for different use cases.

## Development

### Prerequisites
A list of prerequisites for this project and instructions on how to get them.
* [A Github account](https://github.com/)
* [Markdown editor](https://code.visualstudio.com/)
* [Git](https://git-scm.com/)

### Setting up local development environment
Steps needed to setup the local development environment.

Clone repository:
```git clone git@github.com:QOCOSystems/qoco-readme.git```

### Building
Steps needed to build or install the project. Additional information about build tools can be presented.

This project is a simple markdown file, so no additional build steps are required.

### Running or Usage
Steps needed to run or start the project. Additional information about runtime parameters etc. can be presented.

This project can't be run, it is meant to be copied to the root folder of a project in a default branch so that it is the first
thing that developers see when they open the repository. The template acts as a starting point and reference material to create
a project specific README and after copying it should be modified to fit the specific context.

### Deploying
Steps needed to deploy the project to dev/preprod/prod environments.

This project is hosted on GitHub and it is deployed by creating a pull request to master branch (see [Contributing](CONTRIBUTING.md))

## Environments

A list of environments this project is deployed to. Also a description of artifacts or architecture overview can be added.

| Stage | Environment | Artifacts          |
|:------|:------------|:-------------------|
| Prod  | Github      | README.md template |

## Support

Who knows something about this project? Where to find more documentation?

This template was first created as a part of master's thesis by Markus Tyrkk�.

## Related / dependent projects

List of projects that use the API provided by this project for example, or are otherwise dependent or related to this one.
Also a brief description about how they are related to this would be nice to have. An overview of architecture can also be added here.

* [Markdown Cheatsheet](https://github.com/adam-p/markdown-here/wiki/Markdown-Cheatsheet)
  * A cheatsheet for markdown syntax

## Known limitations

A list of known limitations or issues. This might not be necessary as issues could be tracked using other issue tracking methods,
but if there is something that should be mentioned in README already, it could be added here.

	\end{verbatim}
\end{tiny}