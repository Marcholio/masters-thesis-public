\chapter{Conclusions}
\label{chapter:conclusions}

The starting point of this thesis was the case company with an overall feeling that the maintenance process was not working as efficiently as it should be. The company had grown rapidly during the past year
almost doubling its size, which introduced scalability issues on previous ad hoc processes that had been the dominant form of maintenance before adopting a more structured approach in 2017. Despite this,
the process still heavily relied on key personnel, namely the CTO and COO, who both have a long background in the company. The main challenge of the maintenance process was identified to be knowledge sharing,
which caused most of the tasks being assigned to the CTO and COO. As this practice was not scalable and began to slow down new development, it was considered the main challenge to be addressed by this thesis.
In addition to this, there was also a significant amount of maintenance work occurring outside the official process in different informal channels including email and instant messaging applications, which
caused a further difficulty to sharing knowledge on different tasks.

The approach to solve this problem was to evaluate different solutions together with the consultants and developers to encourage participation and co-creation of solutions. As a result of the workshops, it was
agreed that three solutions would be implemented to improve knowledge sharing and the maintenance process in general. A rotating triage responsibility combined with pair working was considered to be the main
solution, which was supported by project presentations during casual Friday meetings and a README template for better quality informal documentation. The triage responsibility was assigned to a group of ten
persons, which covers most of the organization, although not everyone. The main goal for the triage responsible was to ensure that incoming tickets are of sufficient quality and then assign it forward to the
person responsible for solving it, which would improve the overall quality of the tickets and divide the workload of handling newly created tickets more evenly.

The solutions were tested during a seven week implementation phase, after which they were evaluated using the selected metrics and a feedback workshop for data collection. Based on this evaluation, it can be stated
that the solutions managed to address the identified issues, although there is still work to be done in the future. All of the solutions were decided to be continued after the implementation phase with minor
changes, which reflects their suitability to the case company. Especially the main solution of rotating triage responsibility combined with pair working worked quite well and could be applicable to other
similar companies as well.

In addition to the solutions, a key takeaway for the case company is also the methodology for iterative continuous improvement. As there is still a lot of work to be done, this study merely acts as the first
iteration of improvements by raising discussion about various challenges and introducing the methodology of monitoring existing processes and improving them in an iterative manner.
The remaining challenges should be addressed in a similar way in the future to ensure that the well started process does not end up in a standstill.
